In this internship work, we explored the concept of Dynamic Movement Primitives, its properties that make it
attractive in the context of robotics, and the application of this concept to emulate the human behavior via Learning
from Demonstrations. 
We also explored extending DMPS in orientation space using quaternions to enable complete task space motion. 
We also extended DMPs to multiple dimensions such as in planar and task space,
and implemented it on a Kuka LBR iiwa robot in Pybullet simulator.
And lastly, we focused on obstacle avoidance for single and multiple point like objects

The future work for project is to extend this formulation to avoid volumetric obstacles\cite{VolumetricObstacles2019} \cite{Ginesi2021_obstacle}, in an efficient manner.
One can also take multiple demonstrations to imporve upon the learned trajectory of the movement primitive. 

I thank Prof. S.K. Saha and Indian Institute of Technology, Delhi , for the internship oppourtinity and Deepak Raina for the constant support and guidance without
which this project would not have been possible. 




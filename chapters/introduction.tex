The term \textit{Motion Primitives} has its roots in neurobiology and motor control, where researchers
explain the execution of complex motion of biological systems and their ability to adapt to different types of motion effortlessly.
The Dynamic Motion Primitives formulation is one among the many that use the popular approach in \textit{Robot Learning} called Learning from Demonstrations, which uses the demonstrations given by a 
human in some form (teleoperation, kinesthetic control etc\dots) and learn from it to scale to different tasks for robot to perform.
% ! TODO: Add more details and cite the LFD article. 
In this respect, DMPs attempt to answer the question:

\begin{quote}
    \centering
    \textit{How artificial systems can execute
complex movements in a versatile and creative manner?\cite{saveriano2021dynamic}}
\end{quote}

Thus, Dynamic Motion Primitives (DMPs) can be seen as rigorous mathematical formulation of motion primitives as stable nonlinear dyanmical systems.

\section{Dynamic Motion Primitives}

At the heart of DMP lies a spring mass damper system.

\begin{equation}
    \tau \ddot{y} = \alpha \left( -\beta \left( y - y_g \right) - \dot{y} \right) + f(x)
    \label{eq:dmp_equation}
\end{equation}

where, $\alpha$ and $\beta$ are the constants of the spring mass damper system, $y_g$ is the goal state of the system.
This system was first introduced A.J. Ijspeert and his group in \cite{Ijspeert2002} and was further developed in 2013 in \cite{Ijspeert2013}.



